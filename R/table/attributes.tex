% latex.default(属性表, file = "table/attributes.tex", title = "No",      caption = "分析対象の評価表の属性", label = "tab:属性表",      booktabs = T) 
%
\begin{table}[!tbp]
\caption{分析対象の評価表の属性\label{tab:属性表}} 
\begin{center}
\begin{tabular}{llll}
\toprule
\multicolumn{1}{l}{No}&\multicolumn{1}{c}{属性}&\multicolumn{1}{c}{値}&\multicolumn{1}{c}{備考}\tabularnewline
\midrule
1&教員数&10 名&実務家教員\tabularnewline
2&学生数&150 名&約8割が社会人\tabularnewline
3&評価表の数&12 枚&3年×4Q\tabularnewline
4&評価対象数&587 名&評価した学生数\tabularnewline
5&合計文字数&83428 字&総合評価に記述された文字数\tabularnewline
6&平均文字数&142.1 字&合計文字数 $\div$ 評価対象数\tabularnewline
\bottomrule
\end{tabular}
\end{center}
\end{table}

